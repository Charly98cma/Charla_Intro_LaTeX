\begin{frame}{Recap de pequeñas cosas útiles}
    \begin{block}{Creando comandos}
        \texttt{\textbackslash newcommand\{\textbackslash nombre\}[N][M]\ldots{}\{\ldots{}\}}

        \begin{itemize}
            \item[-] \texttt{N}: número de argumentos
            \item[-] \texttt{[M]\ldots{}}\: Valores por defecto de los parámetros
            \item[-] Accediendo a los parámetros: \#X (X = número del parámetro)
        \end{itemize}
    \end{block}

    \pause
    
    \begin{block}{Enlaces con \texttt{hyperref}}
        \texttt{\textbackslash href\{url-mac-url.org\}\{FREE-BEER\}} $\Rightarrow$ \href{https://upm.acm.org/wp/}{ACMUPM}
    \end{block}

    \pause

    \begin{block}{Importando código \LaTeX{} de otro fichero}
        \texttt{\textbackslash include\{path-to-file.tex\}}
    \end{block}
    
\end{frame}


\begin{frame}{Y muuuuuucho más\ldots{}}
    \begin{itemize}
        \item Cabecera y pie de página del documento\pause
        \item Crear clases y paquetes propios\pause
        \item Diagramas (e.g. TikZ)\pause
        \item Presentaciones con \texttt{beamer} (¡como estas diapositivas!) 
        \item Ajedrez! (paquete \texttt{xskak})\pause
        \item y un larguísimo etc\ldots{}
    \end{itemize}
\end{frame}
