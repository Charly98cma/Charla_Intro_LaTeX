\begin{frame}{Comenzando el documento\ldots}

    \begin{block}{\textbackslash documentclass[\textit{ops}]\{\textit{type}\}}
        Define el tipo de documento y sus propiedades (más) básicas
    \end{block}

    \pause

    \vspace{0.5cm}

    \begin{columns}

        \column{0.5\textwidth}
        \textbf{Tipos de documentos}
        \begin{itemize}
            \item \textbf{article}
            \item \textbf{report}
            \item slides/\textbf{beamer}
            \item book
            \item letter
            \item \textit{clase propia}
        \end{itemize}
        
        \pause
        
        \column{0.5\textwidth}
        \textbf{Opciones del documento}
        \begin{itemize}
            \item font size (\textit{10pt}, 12pt, ...)
            \item page size
            \item \texttt{landscape}
            \item \texttt{draft}
            \item \texttt{twoside}
            \item y más\ldots{}
        \end{itemize}
        
    \end{columns}
    
\end{frame}



\begin{frame}{Comenzando el documento\ldots}

    \begin{block}{\textbackslash usepackage[\textit{ops}]\{\textit{pkg-name}\}}
        Sentencia para importar/referenciar paquetes
    \end{block}

    \pause
    \vspace{0.75cm}

    \begin{block}{\textbackslash begin\{document\} \ldots{} \textbackslash end\{document\}}
        Bloque \textit{document} que contiene todo el documento (ya sea en un solo archivo o en múltiples archivos).
    \end{block}

    \pause
    \vspace{0.5cm}

    \begin{center}
        \textit{¡Demo time!}
    \end{center}
\end{frame}